% for notes environment
\usepackage{xsavebox}
\usepackage{hyperref}
\usepackage{graphicx}
\usepackage{luatexja}
\usepackage[hiragino-pro,deluxe,nfssonly,jis2004]{luatexja-preset}
\usepackage{fontspec}
\usepackage{epigraph}
\usepackage{etoolbox}
\usepackage{tikz}
\usepackage{framed}
\usepackage{mathtools}
\usepackage{listings}
\usepackage{libertine}
\usepackage[libertine]{newtxmath}
\usepackage{bxcoloremoji}
\usepackage{xcolor}
\usepackage{diagbox}
\usepackage{caption}
\usepackage{appendixnumberbeamer}
\usepackage{multirow}
\usepackage{xpatch}
\usepackage{multicol}
\usepackage{tabularx}
\usepackage{bussproofs}
\usepackage{plantuml}

\newenvironment{scaledprooftree}[1]%
  {\gdef\scalefactor{#1}\begin{center}\proofSkipAmount \leavevmode}%
  {\scalebox{\scalefactor}{\DisplayProof}\proofSkipAmount \end{center} }

\usetikzlibrary{fit}

\setmonofont{CMU Typewriter Text}

\definecolor{links}{HTML}{2A1B81}
\hypersetup{colorlinks,linkcolor=,urlcolor=links}

\usetheme{Boadilla}
\usecolortheme{seahorse}
% セリフフォント
% \usefonttheme{serif}

\xpatchcmd{\itemize}
  {\def\makelabel}
  {\ifnum\@itemdepth=1\relax
     \setlength\itemsep{1.2ex}% separation for first level
   \else
     \ifnum\@itemdepth=2\relax
       \setlength\itemsep{0.8ex}% separation for second level
       \setlength\topsep{1.2ex}
     \else
       \ifnum\@itemdepth=3\relax
         \setlength\itemsep{0.05ex}% separation for third level
         \setlength\topsep{0.8ex}
   \fi\fi\fi\def\makelabel
  }
 {}
 {}

\setbeamercolor{page number in head/foot}{bg=blue!10}
\setbeamertemplate{footline}{%
  \leavevmode%
  \hbox{%
    \begin{beamercolorbox}[wd=.3\paperwidth,ht=2.25ex,dp=1ex,center]{author in head/foot}%
      \usebeamerfont{author in head/foot}\insertshortauthor\hspace*{1ex}(\insertshortinstitute)
    \end{beamercolorbox}%
    \begin{beamercolorbox}[wd=.3\paperwidth,ht=2.25ex,dp=1ex,center]{title in head/foot}%
      \usebeamerfont{title in head/foot}\insertshorttitle
    \end{beamercolorbox}%
    \begin{beamercolorbox}[wd=.3\paperwidth,ht=2.25ex,dp=1ex,center]{date in head/foot}%
      \insertshortdate{} %@ \InsertConferenceShort
    \end{beamercolorbox}%
    \begin{beamercolorbox}[wd=.1\paperwidth,ht=2.25ex,dp=1ex,center]{page number in head/foot}%
      \insertframenumber{} / \inserttotalframenumber\hspace*{1ex}
    \end{beamercolorbox}}%
  \vskip0pt%
}

\beamertemplatenavigationsymbolsempty

\setbeamertemplate{bibliography item}{\insertbiblabel}
\setbeamersize{description width=1cm}
\setbeamertemplate{items}[circle]
\setbeamertemplate{section in toc}[circle]
\setbeamertemplate{subsection in toc}{%
  \leavevmode\leftskip=2em
  {%
    \usebeamerfont*{itemize item}%
    \usebeamercolor{subsection number projected}%
    \color{bg}%
    \raise1.25pt\hbox{\donotcoloroutermaths$\bullet$}}%
  \hskip1.5ex\inserttocsubsection\par}

% Definitions for the title page
\newcommand*{\GitHub}[1]{%
  \gdef\InsertGitHub{#1}%
}
\newcommand*{\Email}[1]{%
  \gdef\InsertEmail{\href{mailto:#1}{#1}}%
}
\newcommand{\ConferenceImpl}[2][]{%
  \gdef\InsertConferenceShort{#1}%
  \gdef\InsertConference{#2}%
}
\makeatletter
\newcommand\Conference{\@dblarg\ConferenceImpl}
\makeatother
\setbeamerfont{title}{size=\huge, series=\bfseries, family=\mcfamily\rmfamily}
\setbeamercolor{title}{bg=white}
\setbeamerfont{subtitle}{size=\large, series=\mdseries, family=\gtfamily\sffamily}
\setbeamerfont{email}{size=\scriptsize, family=\ttfamily}
\setbeamercolor{email}{bg=white}
\setbeamerfont{date}{shape=\itshape, family=\rmfamily}
\setbeamerfont{vc}{size=\scriptsize, family=\ttfamily}
\setbeamercolor{vc}{bg=white}

\input{vc.tex}

\setbeamertemplate{title page}
{%
  \vbox{}
  \vfill
  \begingroup
    \centering
    \hrulefill\par%
    %\vskip1ex\par%
    \begin{beamercolorbox}[sep=0pt,center,shadow=false,rounded=true]{title}
      \vfill
      \usebeamerfont{title}\inserttitle\par%
      \ifx\insertsubtitle\@empty%
      \else%
        \vskip0.5ex%
        {\usebeamerfont{subtitle}\usebeamercolor[fg]{subtitle}\insertsubtitle\par}%
      \fi%
      \vfill  
    \end{beamercolorbox}%
    \hrulefill\par%
    \vskip1ex%
    \begin{beamercolorbox}[sep=0pt,center,shadow=false,rounded=true]{author}
      \usebeamerfont{author}\insertauthor
    \end{beamercolorbox}
    \begin{beamercolorbox}[sep=0pt,center,shadow=false,rounded=true]{email}
      \usebeamerfont{email}\InsertEmail
    \end{beamercolorbox}
    %\vskip0.1ex
    \begin{beamercolorbox}[sep=5pt,center,shadow=false,rounded=true]{institute}
      \usebeamerfont{institute}\insertinstitute
    \end{beamercolorbox}
    \begin{beamercolorbox}[sep=5pt,center,shadow=false,rounded=true]{date}
      \usebeamerfont{date}\insertdate \normalfont %@ \InsertConference
    \end{beamercolorbox}
    \begin{beamercolorbox}[sep=0pt,center,shadow=false,rounded=true]{vc}
      \usebeamerfont{vc}
      \url{https://github.com/\InsertGitHub} (\texttt{\GITAbrHash})
    \end{beamercolorbox}
    {\usebeamercolor[fg]{titlegraphic}\inserttitlegraphic\par}
  \endgroup
  \vfill
}
\setbeamertemplate{blocks}[rounded][shadow=false]

% ============ ここを消すとNote消える ================
%\mode<handout>{%
%  \usepackage{pgfpages}
%  \setbeameroption{show notes on second screen=right}
%  \setbeamertemplate{note page}{%
%    \vspace{2ex}\insertnote%
%  }
%}
% ============ ここを消すとNote消える ================


\renewcommand{\kanjifamilydefault}{\gtdefault}

\setbeamertemplate{caption}[numbered]
\resetcounteronoverlays{lstlisting}
\definecolor{bluegray}{rgb}{0.4, 0.6, 0.8}
\DeclareCaptionFormat{listing}{{\color{bluegray}\lstlistingname}#2#3}
\captionsetup[lstlisting]{format=listing, font={footnotesize}}
\captionsetup[figure]{name={Fig}}
\captionsetup[table]{name={Table}}
\setbeamerfont{footnote}{size=\scriptsize}

\setmonofont[Ligatures=TeX]{CMU Typewriter Text}

\setbeamertemplate{items}[circle]

\input{./lib/quotebox.tex}
\input{./lib/footnotemark.tex}
\input{./lib/ballon.tex}
\input{./lib/callout.tex}
\input{./lib/listings.tex}
\input{./lib/notes.tex}
\input{./lib/stack.tex}
\input{./lib/card.tex}

\newcommand\ce[1]{%
  \coloremojiucs{#1}
}

\newcommand*{\lstitem}[1]{
  \setbox0\hbox{\lstinline{#1}}
  \item[\usebox0]
}

\presetkeys{todonotes}{inline, noinlinepar}{}

\renewcommand{\arraystretch}{1.2}
\newcolumntype{Y}{>{\centering\arraybackslash}X}

\title{第6章 近隣探索プロトコル}
\subtitle{IPv6勉強会}
\author{吉村優}
\Email{hikaru\_yoshimura@r.recruit.co.jp}
\date{August 4, 2023}
\Conference{}
\institute[\InsertEmail]{}
\GitHub{y-yu/ipv6-slide}

\begin{document}

\frame{%
  \maketitle
  \note{%
    \vspace*{\fill}
    \centering
    {\LARGE Datatype Generic Programming with Scala 3}
    \vspace{2ex}
    
    \begin{itemize}
      \item こっちのページは日本語の説明で、実際の発表では表示されません

      \item ただオーディエンス向けに事前に配っておく資料的な感じで利用する予定です
    \end{itemize}
    \vspace*{\fill}
  }%
}

\section{Introduction}

\begin{frame}
  \frametitle{目次}

  \tableofcontents
\end{frame}

\section{定義}

\begin{frame}
  \frametitle{定義(要約?)(6.1)}

  \begin{itemize}
    \item \textbf{近隣探索プロトコル}としてIPv6本\cite{小川晃通2021-12-20}では次のよう要約されている(RFC 4861)
    \begin{enumerate}
      \item リンク上のルータを探す機能\label{enum:protocol_1}
      \item ルータを経由せずに到達できるIPv6アドレスの範囲を知る機能
      \item リンクのMTUなどの情報を知る機能\label{enum:protocol_3}
      \item インターフェースに対してステートレスにアドレスを振る機能\label{enum:protocol_4}
      \item IPv6アドレスからリンク層のアドレスを解決する機能
      \item 宛先アドレスをもとに、次にパケットを送出すべきIPv6アドレスを知る機能
      \item 近隣ノードに到達できなくなったことを知る近隣不到達性検知
      \item 利用するアドレスが他のノードで使われていないかを確認する機能\label{enum:protocol_8}
      \item ルータからホストへ、より適切な送出先を伝える方法\label{enum:protocol_9}
    \end{enumerate}

    \item \ballref{enum:protocol_1}---\ballref{enum:protocol_3}と\ballref{enum:protocol_9}は
    ルータとホストのやりとりであり、\ballref{enum:protocol_4}---\ballref{enum:protocol_8}はホストとホストのやりとり
  \end{itemize}
\end{frame}

\begin{frame}
  \frametitle{近隣探索プロトコルのモチベーション[独自研究]}

  \begin{itemize}
    \item 宛先IPアドレスがリンク内(同一ルータ配下)かリンク外かで分岐が生じる
    \begin{description}
      \item[リンク内] リンク内にあるノードのMACアドレスを宛先としてリンク層(L2)のパケットをつくって直接送信する
      \item[リンク外] ルータのMACアドレスをリンク層の宛先としてパケットをつくってルータに転送してもらう
    \end{description}
  \end{itemize}

  \begin{columns}
    \begin{column}{0.5\textwidth}
      \begin{center}
        \begin{figure}
          \includegraphics[width=0.99\textwidth]{img/figure1_20.png}
        \end{figure}
      \end{center}
    \end{column}
    \begin{column}{0.5\textwidth}
      \begin{itemize}
        \item よって宛先IPアドレスからMAC(L2アドレス)が解決可能かどうかが重要になる

        \item この近隣探索プロトコルによって、IPアドレスからリンク内のルータやノードのMACアドレスを取得したい
      \end{itemize}
    \end{column}
  \end{columns}
  \begin{itemize}
    \item ついでにMTU(パケットのサイズ?)の設定やリンク内に存在するIPアドレスの範囲も知りたい
  \end{itemize}
\end{frame}

\begin{frame}
  \frametitle{プロトコルで利用するメッセージ(6.1)}

  \begin{itemize}
    \item 近隣探索プロトコルはICMPv6のメッセージ(= IPv6のペイロード)として下記を用いる
    \begin{itemize}
      \item \emph{Router Advertisement} / \emph{Router Solicitation}
      \item \emph{Neighbor Solicitation} / \emph{Neighbor Adovertisement}
      \item Redirect
    \end{itemize}

    \item 近隣探索プロトコルはリンク内で利用されることを前提としているため、
    IPv6のHop Limitを\texttt{255}にしておき、\texttt{255}でなければルータを越えたとして破棄する
    \begin{itemize}
      \item よってICMPv6はIPv6に強く依存しており、このことからL3に分類されるのかもしれない\ce{:thinking:}
    \end{itemize}
  \end{itemize}
\end{frame}

\section{ルータとの近隣探索プロトコル}

\begin{frame}
  \frametitle{%
    ルータとプレフィックス情報\footnote{ルータを経由せずに到達できるIPv6アドレスの範囲のこと。} の発見(6.2)%
  }

  \begin{itemize}
    \item 最初の要約の最初3つだと思う
    \begin{enumerate}
      \item リンク上のルータを探す機能
      \item ルータを経由せずに到達できるIPv6アドレスの範囲を知る機能
      \item リンクのMTUなどの情報を知る機能
    \end{enumerate}

    \item メッセージ``Router Advertisement''と``Router Solicitation''の2つを利用する
  \end{itemize}
\end{frame}

\begin{frame}
  \frametitle{ルータとプレフィックス情報の発見(6.2)}

  \begin{itemize}
    \item ルータはRouter Advertisementメッセージを定期的にノードへマルチキャスト%
    \footnote{マルチキャストでは送信者はネットワークのどこにいてもよく、受信者は複数存在する可能性がある。}する
    \begin{center}
      \begin{figure}
        \includegraphics[width=0.5\textwidth]{img/figure6_1.png}
      \end{figure}
    \end{center}

    \item OptionsのところにMACアドレスなどのデータがはいっている

    \item ノードはRouter Solicitationメッセージをルータへ送信することで、
    Router Advertisementを要求することもできる
  \end{itemize}
\end{frame}

\begin{frame}
  \frametitle{Router Advertisementメッセージ(6.2.1)}
  \begin{center}
    \begin{figure}
      \includegraphics[width=0.65\textwidth]{img/figure6_3.png}
    \end{figure}
  \end{center}

  \begin{itemize}
    \item Optionに含められるのはたとえば次のものがある(RFC 4861, 8106)
    \begin{center}
      \begin{columns}
        \begin{column}{0.4\textwidth}
          \begin{itemize}
            \item Source Link-layer Address
            \item MTU
            \item Prefix Information
          \end{itemize}
        \end{column}
        \begin{column}{0.4\textwidth}
          \begin{itemize}
            \item RDNSS
            \item DNSSL
          \end{itemize}
        \end{column}
      \end{columns}
    \end{center}
  \end{itemize}
\end{frame}

\begin{frame}
  \frametitle{オプションとリンク層アドレス(6.5.2)}

  \begin{itemize}
    \item 下記のようにTLV(Type Length Value)フォーマットでリンク層アドレスを伝搬する
    \begin{center}
      \begin{figure}
        \includegraphics[width=0.65\textwidth]{img/figure6_9.png}
      \end{figure}
    \end{center}
    \begin{description}
      \item[Type] Source Link-layer Addressオプションの場合は1、Target Link-layer Addressオプションの場合は2      
    \end{description}
  \end{itemize}
\end{frame}

\begin{frame}
  \frametitle{ルータ側の処理(6.2.3)}

  \begin{itemize}
    \item デフォルトルータであるかどうかは、Router Lifetimeフィールドを用いて示す

    \item ルータをシャットダウンするなどの理由により送信をやめる場合は、Router Advertisementの
    Router Lifetimeフィールドをゼロに設定して送信するべきとなっている
    
    \item Router AdvertisementメッセージがリンクのMTUを超えてしまう場合、
    オプションを部分的に含んだ複数のRouter Advertisementメッセージに分けて送信することも可能
  \end{itemize}  
\end{frame}

\begin{frame}
  \frametitle{Router Solicitationメッセージ(6.2.2)}

  \begin{itemize}
    \item IPv6ノード側からルータに対してRouter Advertisementをただちに送信するように要求するために用いる
    \begin{center}
      \begin{figure}
        \includegraphics[width=0.55\textwidth]{img/figure6_2.png}
      \end{figure}
    \end{center}

    \item 宛先は通常全ルータへのマルチキャストアドレス\lstinline|ff02::2|となり、
    送信元はリンクローカルアドレスとなる
    \begin{itemize}
      \item IPv6アドレス自動設定でまだIPアドレスがない場合は未定義アドレス\lstinline|::/128|を使う
    \end{itemize}
  \end{itemize}
\end{frame}

\begin{frame}
  \frametitle{Router Solicitationメッセージ(6.2.2)}

  \begin{center}
    \begin{figure}
      \includegraphics[width=0.65\textwidth]{img/figure6_4.png}
    \end{figure}
  \end{center}

  \begin{itemize}
    \item OptionはSource Link-layer Addressのみ
  \end{itemize}
\end{frame}

\begin{frame}
  \frametitle{ホスト側の処理(6.2.4)}

  \begin{itemize}
    \item ルータではないホストがRouter Advertisementメッセージを送信することは禁止

    \item ホストがRouter Solicitationメッセージを受け取った場合には、何もせずに破棄する(RFC 4861)

    \item ホストはルータがデフォルトルータとしての有効期限(Router Lifetime)が過ぎる前に
    新たなRouter Advertisementメッセージによって有効期限を更新しない場合、
    自身の``Default Router List''からそのルータに関するエントリを削除する
  \end{itemize}
\end{frame}

\section{ノード間の近隣探索プロトコル}

\begin{frame}
  \frametitle{リンク層アドレスの解決と近隣不到達性の検知(6.3)}

  \begin{itemize}
    \item IPv4ではARPでやっていたが、IPv6ではICMPv6の
    Neighbor SolicitationメッセージとNeighbor Advertisementメッセージを利用する

    \item Neighbor SolicitationメッセージとNeighbor Advertisementメッセージは、SLAACの
    DAD(Duplicate Address Detection、重複アドレス検知)でも利用される
  \end{itemize}
\end{frame}

\begin{frame}
  \frametitle{Solicited-Nodeマルチキャストアドレスグループ(6.3.3)}

  \begin{itemize}
    \item ``Solicited-Nodeマルチキャストアドレス''は、
    特定のIPv6アドレスに関連付けられたリンク層アドレスを決定するために近隣探索プロトコルによって使用されるIPv6マルチキャストアドレス

    \item おそらく\textbf{Solicited-Nodeマルチキャストアドレスグループ}はリンク内の全てのノードになるのか?\ce{:thinking:}
  \end{itemize}
\end{frame}

\begin{frame}
  \frametitle{Neighbor Solicitation/Advertisementメッセージ(6.3.1, 2)}

  \begin{itemize}
    \item Neighbor Solicitationメッセージは、リンク層アドレスの解決を行う場合にはSolicited-Nodeマルチキャストで送信
    \begin{itemize}
      \item Optionにリンク層アドレスが入っている
    \end{itemize}

    \item Neighbor Solicitationを受けとったノードは自分のリンク層アドレスをNeighbor Advertisementメッセージで返答する
  \end{itemize}
\end{frame}

\begin{frame}[fragile]
  \frametitle{近隣キャッシュ(6.3.4)}

  \begin{columns}
    \begin{column}{0.7\textwidth}
      \begin{minipage}{\columnwidth}
        \begin{plantuml}
          @startuml

          actor NodeA AS "Node A"
          actor NodeB AS "Node B"
          actor NodeC AS "Node C"

          autonumber

          NodeA -> NodeA: パケットを送りたいNodeBのIPv6アドレスのリンク層アドレスが分かるか?
          alt NodeBのリンク層アドレスが近隣キャッシュにある
            NodeA -> NodeB: パケット送信
          else わからない
            NodeA -> NodeB: Neighbor Solicitationを送信
            NodeA -> NodeC: Neighbor Solicitationを送信
            NodeA -> NodeA: 近隣キャッシュにNodeBのIPアドレスをIMCOMPLETEで保存
            NodeB --> NodeA: Neighbor Advertisementを返信
            NodeC --> NodeA: Neighbor Advertisementを返信

            alt Neighbor Advertisementの送信元IPアドレスがNodeBである
              NodeA -> NodeA: 近隣キャッシュのNodeBのリンク層アドレスを更新
            else 異なる
              NodeA -> NodeA: パケットを破棄
            end
          end
          @enduml
        \end{plantuml}
      \end{minipage}
    \end{column}
    \begin{column}{0.3\textwidth}
      \begin{itemize}
        \item 近隣キャッシュはIPv6アドレスとリンク層アドレスの辞書

        \item 返ってきたNeighbor Advertisementのうち送信元が狙ったIPアドレスだけを
        近隣キャッシュ更新に用いる
      \end{itemize}
    \end{column}
  \end{columns}
\end{frame}

\begin{frame}
  \frametitle{近隣不到達性検知(6.3.5)}
 
  \begin{itemize}
    \item 本では\ce{:point_down:}のようにあるが、これは到達可能性のはなし?\ce{:thinking:}
    \begin{itemize}
      \item 近隣不到達性検知では、自ノードから送ったIPv6パケットを受け取ったことを知らせる何らかの通知を相手ノードから受け取ったときに、その相手ノードへは到達可能であると判断します
    \end{itemize}

    \item ようするにNeighbor Solicitationを送ったけど、狙ったIPアドレスから返答がなかったということ?
  \end{itemize}
\end{frame}

\begin{frame}
  \frametitle{近隣キャッシュのステート(6.3.6)}
  
  \begin{itemize}
    \item いろいろある
  \end{itemize}
\end{frame}

\section{Redirect}

\begin{frame}
  \frametitle{Redirectメッセージ(6.4)}

  \begin{itemize}
    \item Redirectメッセージはルータからホストへ次のようなときに送信される
    \begin{itemize}
      \item 他のルータが特定の経路としてより良い次ホップであることをルータからホストに対して伝えるとき
      \item 特定の宛先が実際には同一リンク上に接続された近隣ノードであることを伝えるとき
    \end{itemize}

    \item IPv6はリンク内に複数のルータがあるから、そういうこともある?\ce{:thinking:}
    \begin{center}
      \begin{figure}
        \includegraphics[width=0.65\textwidth]{img/figure6_16.png}
      \end{figure}
    \end{center}
  \end{itemize}
\end{frame}

\section{on-linkとoff-link}

\begin{frame}
  \frametitle{on-linkとoff-link(6.6)}

  \begin{center}
    \begin{figure}
      \includegraphics[width=0.55\textwidth]{img/figure6_16.png}
    \end{figure}
  \end{center}

  \begin{itemize}
    \item 同一リンクに接続されていることをon-link、同一リンクに接続されていないことをoff-linkと表現する

    \item IPv6ではIPアドレスとプレフィックスによりon-linkかどうかを決定できない
    \begin{itemize}
      \item IPv4はできたが、IPv6は1つのネットワークインターフェースに複数のIPv6アドレスが設定できるためできない
    \end{itemize}

    \item たとえば上の図であれば、プレフィックスは異なるがホストAからホストBはon-linkである
    %\begin{itemize}
    %  \item 別プレフィックスでoff-linkと判断すると、ホストA→ホストBはルータ2を経由する必要がある
    %\end{itemize}
  \end{itemize}
\end{frame}

\begin{frame}
  \frametitle{on-link情報(6.6.1)}

  \begin{itemize}
    \item Router Advertisementにあるプレフィックス情報を使う

    \item Redirectメッセージが示すTarget Addressがリンクローカルアドレスではなく、
    かつDestination AddressがTarget Addressと同じものであるとき、
    そのメッセージはTarget Addressで示される宛先がon-linkであることを示す
  \end{itemize}
\end{frame}

% \begin{frame}
%   \frametitle{Who am I\ce{:question:}}
  
%   \begin{columns}
%     \begin{column}{0.3\textwidth}
%       \begin{center}
%         \begin{figure}
%           \includegraphics[width=0.95\textwidth]{img/bird2x.png}
%         \end{figure}
%       \end{center}
 
%       \begin{table}[h]
%         \begin{tabular}{ll}
%           Twitter & \href{https://twitter.com/\_yyu\_}{@\_yyu\_} \\
%           Qiita &  \href{https://qiita.com/yyu}{yyu} \\
%           GitHub &  \href{https://github.com/y-yu}{y-yu} \\
%         \end{tabular}
%       \end{table}
%     \end{column}
%     \begin{column}{0.7\textwidth}
%       \begin{itemize}
%         \item Recruit Co., Ltd.
%         \begin{itemize}
%           \item StudySapuri ENGLISH server side
%         \end{itemize}

%         \item Quantum Information \& Algorithms

%         \item Cryptography \& Security
        
%         \item Programming \& {\LaTeX} typesetting
%         \begin{itemize}
%           \item Scala, Rust, Go, Swift
%         \end{itemize}
%       \end{itemize}
%     \end{column}
%   \end{columns}

%   \note{%
%     \begin{itemize}
%       \item スタディサプリENGLISHをやっている

%       \item 量子コンピュータに興味があったり、セキュリティーや暗号もやってたりする

%       \item Rustなど他のプログラム言語や、この資料を作るのにも使っている{\LaTeX}も興味がある
%     \end{itemize}
%   }
% \end{frame}

% \begin{frame}[fragile]
%   \frametitle{\lstinline|TestObject|: generating fixtures for unit tests}

%   \begin{itemize}
%     \item<+-> We use \lstinline|TestObject| on our product,
%     which is a utility for generating dummy objects(also known as \emph{fixtures}) for unit tests.
% \begin{lstlisting}[style=scala]
% case class StudySapuriSession(
%   /* very complicated! */
% )
% val dummyData = TestObject.get[StudySapuriSession]
% \end{lstlisting}

%     \item<+-> Type class \lstinline|TestObject[A]| provides us with a way to generate some value of \lstinline|A|.
%     \begin{columns}
%       \begin{column}{0.32\textwidth}
% \begin{lstlisting}[style=scala]
% trait TestObject[A] {
%   def generate: State[Int, A]
% }
% \end{lstlisting}
%       \end{column}
%       \begin{column}{0.68\textwidth}
% \begin{lstlisting}[style=scala]
% implicit val strInstance: TestObject[String] = new TestObject {
%   def generate: State[Int, A] = State(s => (s + 1, s.toString))
% }
% \end{lstlisting}
%       \end{column}
%     \end{columns}

%     \item<+-> In a naive way, we would have to define too many \lstinline|TestObject| implicit instances for
%     every type used in our product, but it's not possible or reasonable.
%   \end{itemize}

%   \note{
%     \begin{itemize}
%       \item このトークではこの\lstinline|TestObject|の話をする

%       \item これは単体テストなどのモッキング用に、任意の型\lstinline|A|のダミー値を作成することができる

%       \item 実態としてはステートモナドになっており、この\lstinline|Int|のステートを元にして型\lstinline|A|の値を作成する

%       \item ナーイブには、このような\lstinline|implicit|インスタンスを、プロダクトに存在する型のぶんだけ
%       大量に定義していく必要があるが、そのような作業は無理である
%     \end{itemize}
%   }
% \end{frame}

% \section{Overview of datatype generic programming}

% \begin{frame}
%   \frametitle{\lstinline|TestObject| and datatype generic programming}

%   \pause
%   \begin{itemize}
%     \item<+-> The many \lstinline|TestObject| instances can be provided by
%     \emph{datatype generic programming}, rather than manually.
%     \begin{itemize}
%       \item We can easily obtain \lstinline|dummyData: StudySapuriSession| once we define
%       \lstinline|TestObject| instances for primitive or Java types,
%       \item And then datatype generic programming generates the other instances for our
%       defined data structures(= case objects).
%     \end{itemize}

%     \item<+-> Datatype generic programming is one of the ways of meta-programming.

%     \item<+-> In this talk I'll explain datatype generic programming with Scala 3.
%   \end{itemize}

%   \note{
%     \begin{itemize}
%       \item このようなときに\emph{datatype generic programming}を使うことができる。
%       手作業でのインスタンス定義を回避して自動的にインスタンスを提供させる
%       \begin{itemize}
%         \item ユーザーが手でやるのはプリミティブな型やJavaの型だけでよい
%       \end{itemize}

%       \item Datatype generic programmingはメタプロの一種である

%       \item しかしdatatype generic programmingのやり方がScala 2と3で相当変わってしまった。
%       今回のトークではScala 3でのやり方について解説する
%     \end{itemize}
%   }
% \end{frame}

% \begin{frame}[fragile]
%   \frametitle{Datatype generic programming \textit{vs.} raw macros}

%   \pause
%   \begin{itemize}
%     \item<+-> Unfortunately if we were to use raw macros, we could arbitrarily edit syntax trees,
%     but we don't want that.
%     \begin{itemize}
%       \item For instance, it's difficult for us to write \emph{sbt} settings even if we know Scala\ce{:innocent:}
%     \end{itemize}

%     \item<+-> Almost every data structure can be classified as either ``tuple''-like or ``enum''-like:
%     \begin{columns}
%       \begin{column}{0.32\textwidth}
% \begin{lstlisting}[style=scala]
% case class TupleLike(
%   field1: Int, field2: String
% )
% \end{lstlisting}
%       \end{column}
%       \begin{column}{0.5\textwidth}
% \begin{lstlisting}[style=scala]
% sealed trait EnumLike
% case class Pattern1(v: Int)    extends EnumLike
% case class Pattern2(v: String) extends EnumLike
% \end{lstlisting}
%       \end{column}
%     \end{columns}
%     \begin{itemize}
%       \item \lstinline|TupleLike| requires both two values of \lstinline|Int| and \lstinline|String|,
%       whereas \lstinline|EnumLike| requires either an \lstinline|Int| value or a \lstinline|String| value.
%     \end{itemize}

%     \item<+-> Datatype generic programming provides us with the following two functions: 
%       \ballcircle{1} converting a value to the analogy tuple or enum
%       \ballcircle{2} and reverting it to the original type.
%   \end{itemize}

%   \note{
%     \begin{itemize}
%       \item 我々が生のマクロを使うと、シンタックスツリーを任意に変更できてしまうが、それはあまり望んでいない
%       \begin{itemize}
%         \item たとえばScalaを知っていても\emph{sbt}の設定を書くのが難しい
%       \end{itemize}

%       \item ほぼ(?)すべてのデータ構造はタプルのような構造と、enumのような構造のどちらかである
%       \begin{itemize}
%         \item タプルは含まれている型の全ての値が必要で、一方でenumは含まれる型のうちどれかしら1つを要請する
%         \item この例だと\lstinline|TupleLike|は\lstinline|Int|と\lstinline|String|の両方が必要だが、
%         一方で\lstinline|EnumLike|はどちからがあればインスタンシエイトできる
%       \end{itemize}

%       \item こういう感じで全てのデータ構造についてタプルとenumに分けていき、
%       そのタプルを操作して再び元のデータ構造に戻すみたいな抽象化を提供するのがDatatype generic programmingである
%     \end{itemize}
%   }
% \end{frame}

% \begin{frame}[fragile]
%   \frametitle{Meta-programming using datatype generic programming}
  
%   \begin{itemize}
%     \item Almost all meta-programming can be done using such datatype abstraction
%     and ad-hoc polymorphism, without the need to edit syntax trees.
%     \pause
%     \begin{enumerate}
%       \item First, convert user-defined data structures(= case objects) to tuple or enum like using datatype generic programming.
%       \item Then, find some implicit instances based on the types included in the tuple or enum.
%       \item Finally, revert the derived instance for tuple or enum like to one for the original data type.
%     \end{enumerate}

%     \pause
%     \item \lstinline|case class TupleLike(field1: Int, field2: String)| example:
%     \begin{scaledprooftree}{0.8}
%       \AxiomC{\texttt{TupleLike $\Leftrightarrow$ (Int, String)}}
%       \AxiomC{\lstinline|TestObject[Int]|}
%       \AxiomC{\lstinline|TestObject[String]|}
%       \RightLabel{{\small \ballcircle{2}}}
%       \BinaryInfC{\lstinline|TestObject[(Int, String)]|}
%       \RightLabel{{\small \ballcircle{1}}}
%       \BinaryInfC{\lstinline|TestObject[TupleLike]|}
%     \end{scaledprooftree}
%     where {\small \texttt{TupleLike $\Leftrightarrow$ (Int, String)}} is powered by datatype generic programming.
%   \end{itemize}

%   \note{
%     \begin{itemize}
%       \item このようにタプルやenumとデータ構造を行きするような抽象化と、アドホック多相があればほぼ全てのメタプログラミングを
%       シンタックスツリーの変更なしで行うことができる
%       \begin{enumerate}
%         \item datatype generic programmingでデータ構造をタプルやenumに変換する
%         \item そのタプルやenumに含まれている型のimplicit インスタンスを探索する
%         \item そして元々の型に復活させる
%       \end{enumerate}

%       \item 具体的に\lstinline|TupleLike|で見ていくとこういう感じになる
%     \end{itemize}
%   }
% \end{frame}

% \begin{frame}
%   \frametitle{Macro compatibility between Scala 2 and 3}

%   \begin{itemize}
%     \item We use \emph{shapeless}\cite{shapeless_github} for datatype generic programming in Scala 2.

%     \pause
%     \item In our product, we compile \& test both of Scala 2 and 3 for almost all of our code.

%     \pause
%     \item There is no compatibility of macros between Scala 2 and 3 \ce{:innocent:}

%     \pause
%     \item It follows that \lstinline|TestObject| implemented on Scala 2 won't work well in Scala 3.
%     \begin{itemize}
%       \item \emph{shapeless 3}\cite{shapeless-3_github} for Scala 3 is being developed but unfortunately
%       it doesn't have compatibility of shapeless for Scala 2\ce{:innocent:}
%     \end{itemize}

%     \pause
%     \begin{columns}
%       \begin{column}{0.6\textwidth}      
%         \item Eventually we (mainly \emph{ScalaNinja}) began to develop another \lstinline|TestObject| implementation for Scala 3
%       \end{column}
%       \begin{column}{0.2\textwidth}
%         \begin{figure}[h]
%           \includegraphics[width=0.6\columnwidth]{img/xuwei.png}
%           \caption{ScalaNinja}
%           \label{fig:scalaninja}
%         \end{figure}
%       \end{column}
%     \end{columns}
%   \end{itemize}

%   \note{
%     \begin{itemize}
%       \item 我々はScala 2でのdatatype generic programmingにshapelessをつかっている

%       \item ところで、我々のプロダクトはScala 3でもほぼ全てのコードをコンパイル\&テストしている

%       \item そしてScala 2と3でマクロの互換性がない

%       \item つまりScala2で作ってあった\lstinline|TestObject|はScala 3では動かない。
%       Scala 3対応したshapeless 3も作らてはいるが、これはshapeless 2とは別物というくらいに互換性がない

%       \item そういうわけでScala 3版の\lstinline|TestObject|を作ることになった
%     \end{itemize}
%   }
% \end{frame}

% \section{Datatype generic programming in Scala 3}

% \begin{frame}[fragile]
%   \frametitle{Datatype generic programming in Scala 3}

%   \pause
%   \begin{itemize}
%     \item<+-> Scala 3 supports datatype generic programming initially.
    
%     \item<+-> Some functions can be used to convert case objects from/to tuple like without any libraries like follows:
%     \begin{columns}
%       \begin{column}{0.32\textwidth}
% \begin{lstlisting}[style=scala]
% import scala.compiletime.*
% import scala.deriving.*
% case class TupleLike(
%   field1: Int, field2: String
% )
% \end{lstlisting}
%       \end{column}
%       \begin{column}{0.6\textwidth}
% \begin{lstlisting}[style=scala]
% scala> Tuple.fromProductTyped(TupleLike(1, "a"))
% val res0: (Int, String) = (1,a)

% scala> summon[Mirror.ProductOf[TupleLike]].fromProduct(res0)
% val res1: TupleLike = TupleLike(1,a)
% \end{lstlisting}
%       \end{column}
%     \end{columns}
%     \begin{itemize}
%       \item Similarly some functions can be used to convert \lstinline|sealed trait| to enum like.
%     \end{itemize}

%     \item<+-> Meta-programming tools in Scala 3 is reinforced rather than Scala 2\ce{:thumbsup:}
%   \end{itemize}

%   \note{
%     \begin{itemize}
%       \item 実はScala 3はライブラリーなしでこのようにケースクラスをタプルに変換するなどの機構が用意されている

%       \item こんな感じで\lstinline|scala.compiletime|や\lstinline|scala.deriving|を引っ張ってくることで、
%       任意のケースクラスをそのフィールドの型に対応するタプルにしたり、タプルから構成したりできる
      
%       \item 今回は紹介しない部分も含めて、Scala 3はメタプログラミングがScala 2に比べて強化されている
%     \end{itemize}
%   }
% \end{frame}

% \begin{frame}[fragile]
%   \frametitle{\lstinline|TestObject| implementation on Scala 3}

%   \begin{itemize}
%     \item<+-> We'll define \lstinline|derive| method as the final goal, like this:
% \begin{lstlisting}[style=scala]
% inline implicit def derive[A]: TestObject[A]
% \end{lstlisting}   
%     \begin{itemize}
%       \item \lstinline|derive| provides \lstinline|TestObject| instances for all \lstinline|A|.
%     \end{itemize}

%     \item<+-> This is an overview of the \lstinline|derive| behavior:
%     \begin{enumerate}
%       \item Check if the instance for the input type has been defined.
%       \item If not found, pattern match the type into either tuple like or enum like. \label{enum:tuple_or_enum}
%       \item Collect the \emph{ill-typed} list of \lstinline|TestObject| for each types contained in \ballref{enum:tuple_or_enum}
%       using \lstinline|erasedValue| \label{enum:instances_list}.
%       \item Finally, make the instance for the input type using \ballref{enum:instances_list} instances list.
%     \end{enumerate}

%     \item<+-> Let's see the details!
%   \end{itemize}

%   \note{
%     \begin{itemize}
%       \item 今回の目標はScala 3のメタプロ機構で任意の型\lstinline|A|に対する\lstinline|TestObject[A]|を提供する
%       \lstinline|derive|メソッドをつくること

%       \item \lstinline|derive|は
%       \begin{enumerate}
%         \item すでに定義されているインスタンスがないか探して
%         \item もしなかったら、タプルかenumのケースへパターンマッチする
%         \item \lstinline|erasedValue|をつかって、上記のタプルかenumに入っている型のインスタンスを集めてきて
%         型なしリストに詰め込む
%         \item そして最後に\ce{:point_up:}のリストで材料をつくって最終的なインプットされた型のインスタンスをつくる
%       \end{enumerate}
%     \end{itemize}
%   }
% \end{frame}

% \begin{frame}[fragile]
%   \frametitle{\ballcircle{1} Check if the instance for the input type has been defined}

%   \begin{itemize}
%     \item<+-> \lstinline|summonFrom| searches the \lstinline|TestObject| instance for type \lstinline|A|.
%     \begin{columns}
%       \begin{column}{0.5\textwidth}
% \begin{lstlisting}[style=scala]
% inline implicit def derive[A]: TestObject[A] =
%   summonFrom {
%     case x: TestObject[A] =>
%       x
%     case _ =>
%       create[A] // we'll define next page!
%   }
% \end{lstlisting}
%       \end{column}
%       \begin{column}{0.3\textwidth}
%         \begin{figure}[h]
%           \includegraphics[width=0.5\columnwidth]{./img/fantasy_mahoujin_syoukan.png}
%           \caption{Image of \lstinline|summonFrom|}
%         \end{figure}
%       \end{column}
%     \end{columns}


%     \item<+-> If \lstinline|summonFrom| finds a \lstinline|TestObject[A]| instance,
%     then the instance will be assigned to the variable \lstinline|x|.
%     \begin{itemize}
%       \item In this case, it's unnecessary to define a new instance so \lstinline|derive| returns \lstinline|x|.
%     \end{itemize}

%     \item<+-> In the latter case, we call \lstinline|create| method to define \lstinline|TestObject[A]|.
%   \end{itemize}

%   \note{
%     \begin{itemize}
%       \item \lstinline|summonFrom|をつかってインスタンスを探すことができる

%       \item もし見つかったらそれを返せばいい

%       \item そうでない場合、\lstinline|create|メソッドをつかって定義していく
%     \end{itemize}
%   }
% \end{frame}

% \begin{frame}[fragile]
%   \frametitle{\ballcircle{2} Pattern matching if \lstinline|A| is \lstinline|ProductOf[A]| or \lstinline|SumOf[A]|}

%   \begin{itemize}
%     \item<+-> Since there is no \lstinline|TestInstance[A]| instance yet,
%     \lstinline|create| finds \lstinline|ProductOf[A]| or \lstinline|SumOf[A]|
%     instance using \lstinline|summonFrom| again.
% \begin{lstlisting}[style=scala]
% inline final def create[A]: TestObject[A] =
%   summonFrom {
%     case _: Mirror.ProductOf[A] =>
%       deriveProduct[A] // 1
%     case _: Mirror.SumOf[A] =>
%       deriveSum[A]     // 2
%   }
% \end{lstlisting}

%     \item<+-> This means that:
%     \begin{enumerate}
%       \item \lstinline|A| is a tuple-like type (i.e. case classes)
%       if there is a \lstinline|ProductOf[A]| instance,
%       \item \lstinline|A| is an enum-like structure (i.e. sealed traits).
%       if there is a \lstinline|SumOf[A]| instance.
%     \end{enumerate}
%   \end{itemize}

%   \note{
%     \begin{itemize}
%       \item \lstinline|TestInstance[A]|が見つからなかったので、
%       \lstinline|create|では再び\lstinline|summonFrom|を使って
%       \lstinline|ProductOf[A]|か\lstinline|SumOf[A]|のインスタンスを探す
%       \begin{itemize}
%         \item もし\lstinline|ProductOf[A]|が見つかったら、型\lstinline|A|は
%       ケースクラスなどのタプル的なデータ構造であり、
%         \item 一方で\lstinline|SumOf[A]|のインスタンスが見つかれば、
%         型\lstinline|A|はsealed traitなどのenum的なデータ構造である
%       \end{itemize}
%     \end{itemize}
%   }
% \end{frame}

% \begin{frame}[fragile]
%   \frametitle{\ballcircle{3} Making a list \lstinline|List[TestObject[?]]| of \emph{ill-typed} instances}
  
%   \begin{itemize}
%     \item Before seeing \lstinline|deriveProduct| and \lstinline|deriveSum|,
%     we need to prepare the way to collect all instances for types being contained in \lstinline|A|.
%     \begin{columns}
%       \begin{column}{0.55\textwidth}
%         \begin{itemize}
%           \item For example \lstinline|TupleLike|, 
%           we need the both instances of \lstinline|TestObject[Int]| and \lstinline|TestObject[String]|.
%         \end{itemize}
%       \end{column}
%       \begin{column}{0.35\textwidth}
% \begin{lstlisting}[style=scala]
% case class TupleLike(
%   field1: Int, field2: String
% )
% \end{lstlisting}
%       \end{column}
%     \end{columns}

%     \pause
%     \item \lstinline|erasedValue| allows us to search and collect all instances recursively as follows:
%     \begin{columns}
%       \begin{column}{0.6\textwidth}
% \begin{lstlisting}[style=scala]
% inline def deriveRec[T <: Tuple]: List[TestObject[?]] =
%   inline erasedValue[T] match {
%     case _: EmptyTuple =>
%       Nil
%     case _: (t *: ts) =>
%       derive[t]/* mutual recursion */ :: deriveRec[ts]
%   }
% \end{lstlisting}
%       \end{column}
%       \begin{column}{0.3\textwidth}
%         \item There is no type compatibility among the instances,
%         \lstinline|deriveRec| cannot help but to return \emph{ill-typed} list\ce{:innocent:}
%       \end{column}
%     \end{columns}

%     \pause
%     \item Additionally, \lstinline|*:| is type-level tuple constructor
%     provided since Scala 3.
%   \end{itemize}

%   \note{
%     \begin{itemize}
%       \item  \lstinline|deriveProduct|とか\lstinline|deriveSum|を見ていくまえに
%       タプルとかenumに所属する型のインスタンスを全部あつめてくる\lstinline|deriveRec|を準備しておく

%       \item たとえば\lstinline|TupleLike|なら
%       \lstinline|TestObject[Int]|と\lstinline|TestObject[String]|の両方のインスタンスを集めてくる

%       \item \lstinline|erasedValue|を使うと再帰的に集めてくることができる

%       \item ただし\lstinline|TestObject[Int]|と\lstinline|TestObject[String]|みたいに
%       インスタンスの間に型の互換性はないから、これを1つのリストに詰め込むと型が壊れる

%       \item ちなみに\lstinline|*:|は型レベルタプルで、\lstinline|HList|みたいなもの
%     \end{itemize}
%   }
% \end{frame}

% \begin{frame}[fragile]
%   \frametitle{\ballcircle{4a} In \lstinline|deriveProduct| case}
  
%   \begin{itemize}
%     \item<+-> Using \lstinline|deriveRec|, we define a \lstinline|TestObject| instance for \lstinline|A|
%     in \lstinline|deriveProduct|.
% \begin{lstlisting}[style=scala]
% inline def deriveProduct[A](using a: ProductOf[A]): TestObject[A] = {
%   def p: TestObject[A] = {
%     val xs = deriveRec[a.MirroredElemTypes] // `a.MirroredElemTypes` is analogy tuple of `A`.
%     productImpl[A](xs, a)
%   }
%   p
% }
% \end{lstlisting}
%     \begin{itemize}
%       \item Why does \lstinline|deriveProduct| only call \lstinline|productImpl|
%       through a temporary method \lstinline|p|\ce{:thinking:}
%     \end{itemize}

%     \item<+-> This is ScalaNinja's remarkable and state-of-the-art technique to avoid:
%     \begin{itemize}
%       \item throwing \lstinline|MethodTooLargeException| due to \lstinline|inline|
%       \item and generating too many nameless classes.
%     \end{itemize}

%     \item<+-> In meta-programming, we have to also consider compiling efficiency, not only runtime.
%     That's maybe the why meta-programming is difficult\ce{:innocent:}
%   \end{itemize}

%   \note{
%     \begin{itemize}
%       \item \lstinline|deriveRec|をつかって\lstinline|TestObject[A]|のインスタンスを作っていく

%       \item これは謎にメソッド\lstinline|p|の中で\lstinline|productImpl|呼ぶだけとなっているのは、
%       実は
%       \begin{itemize}
%         \item \lstinline|inline|のコード生成で\lstinline|MethodTooLargeException|がブチあがるのを避け、
%         \item 無名クラスではなくてメソッドにしておくことで、無名クラスの大量生成も避ける
%       \end{itemize}
%       というScalaNinjaのテクニックとなっている

%       \item こういうコンパイル時の効率も考慮しないといけないので、メタプロは難しい
%     \end{itemize}
%   }
% \end{frame}

% \begin{frame}[fragile]
%   \frametitle{\ballcircle{4a} In \lstinline|deriveProduct| case}

%   \begin{itemize}
%     \item<+-> First, we create a tuple naming \lstinline|values| which are containing all values 
%     required by \lstinline|A|.
% \begin{lstlisting}[style=scala]
% final def productImpl[A](xs: List[TestObject[?]], a: ProductOf[A]): TestObject[A] =
%   new TestObject[A] {
%     def generate: IntState[A] =
%       for {
%         values <- xs.traverse(_.generate.widen[Any])
%       } yield a.fromProduct(new SeqProduct(values))
%   }
% \end{lstlisting}
%     \begin{itemize}
%       \item It's important that \lstinline|productImpl| doesn't have \lstinline|inline|.
%     \end{itemize}

%     \item<+-> Then, we create a \lstinline|A| value using \lstinline|a.fromProduct|.
%   \end{itemize}

%   \note{
%     \begin{itemize}
%       \item \lstinline|List[TestObject[?]]|をつかってまず必要な値をつくる

%       \item それを\lstinline|a.fromProduct|で型\lstinline|A|に戻す

%       \item ここで\lstinline|productImpl|には\lstinline|inline|がないのが地味に重要
%       \begin{itemize}
%         \item こうしておくことで\lstinline|inline|でメソッドが巨大化しすぎてエラーとなるのを回避する
%       \end{itemize}
%     \end{itemize}
%   }
% \end{frame}

% \begin{frame}[fragile]
%   \frametitle{\ballcircle{4b} In \lstinline|deriveSum| case}

%   \begin{itemize}
%     \item In \lstinline|SumOf| case, we generate a value in \lstinline|values|.
% \begin{lstlisting}[style=scala],
% inline def deriveSum[A](using a: SumOf[A]): TestObject[A] = {
%   def s: TestObject[A] = {
%     val values = deriveRec[a.MirroredElemTypes]
%     sumImpl[A](values)
%   }
%   s
% }
% \end{lstlisting}
%     \item It's very similar to \lstinline|deriveProduct|.
%   \end{itemize}

%   \note{
%     \begin{itemize}
%       \item \lstinline|SumOf|の場合を\lstinline|deriveProduct|と同じようにできる
%     \end{itemize}
%   }
% \end{frame}

% \begin{frame}[fragile]
%   \frametitle{\ballcircle{4b} In \lstinline|deriveSum| case}

%   \begin{itemize}
%     \item<+-> \lstinline|sumImpl| is a very complicated function\ce{:innocent}
% \begin{lstlisting}[style=scala]
% final def sumImpl[A](values: List[TestObject[?]]): TestObject[A] =
%   new TestObject[A] {
%     def generate: IntState[A] =
%       for {
%         allResults <- values.traverse(_.generate.widen[Any])
%         l = allResults.minBy(_.getClass.getName)
%         rOpt = allResults.tail.headOption.flatMap(
%           _ => allResults.maxByOption(_.getClass.getName)
%         )
%         s <- State.get
%       } yield rOpt match {
%         case Some(r) => if (s % 2 == 0) l.asInstanceOf[A] else r.asInstanceOf[A]
%         case None => l.asInstanceOf[A]
%       }
%   }
% \end{lstlisting}

%     \item<+-> What's the purpose of \lstinline|minBy(_.getClass.getName)| and \lstinline|maxByOption(_.getClass.getName)|?
%   \end{itemize}

%   \note{
%     \begin{itemize}
%       \item \lstinline|sumImpl|はめちゃくちゃ複雑

%       \item だけど実装は適当で、偶数か奇数かによってどれを選んでるくるか?みたいなことをしているだけ

%       \item ところで\lstinline|minBy(_.getClass.getName)|と\lstinline|maxByOption(_.getClass.getName)|はなんなのか?
%     \end{itemize}
%   }
% \end{frame}

% \begin{frame}[fragile]
%   \frametitle{\ballcircle{4b} In \lstinline|deriveSum| case}

%   \begin{itemize}
%     \item<+-> I don't know why, but anyway shapeless 2 sorts the instances by their type name\cite{shapeless2_sort}.
  
%     \item<+-> On the other hand, \lstinline|MirroredElemTypes| in Scala 3 are sorted by the definitions in source code.

%     \item<+-> For instance:
%     \begin{columns}
%       \begin{column}{0.28\textwidth}
% \begin{lstlisting}[style=scala]
% sealed trait X
% case object X3 extends X
% case object X2 extends X
% case object X1 extends X
% \end{lstlisting}
%       \end{column}
%       \begin{column}{0.57\textwidth}
%         \begin{itemize}
%           \item shapeless 2 returns \lstinline|X1 :+: X2 :+: X3|, which is sorted by alphabetical order,
%           \item but \lstinline|MirroredElemTypes| is \lstinline|X3 *: X2 *: X1|\ce{:innocent:}
%         \end{itemize}
%       \end{column}
%     \end{columns}

%     \item<+-> So the \lstinline|minBy| and \lstinline|maxByOption| are needed for
%     the compatibility of shapeless 2 behavior.
%   \end{itemize}

%   \note{
%     \begin{itemize}
%       \item 理由は不明だがshapeless 2は見つかったインスタンスをその名前順にソートして返すようになっている

%       \item しかしScala 3の\lstinline|MirroredElemTypes|は定義した順に返ってくる

%       \item そこらの互換性のために一応順序をいじっている
%     \end{itemize}
%   }
% \end{frame}

% \section{Conclusion}

% \begin{frame}
%   \frametitle{Conclusion}

%   \pause
%   \begin{itemize}
%     \item<+-> The full source code for this talk has been published at \ce{:point_down:}
%     \begin{itemize}
%       \item \url{https://github.com/y-yu/test-object}
%       \item It may be useful as proof-of-concept to compare Scala 2(shapeless 2) with Scala 3.
%     \end{itemize}

%     \item<+-> There is no macro compatibility between Scala 3 and 2\ce{:innocent:}
%     \begin{itemize}
%       \item And shapeless 2 and 3 don't have the same interface.
%     \end{itemize}

%     \item<+-> Scala 3 supports datatype generic programming initially.
%     \begin{itemize}
%       \item Is there any ways how not using ill-typed list?\ce{:thinking:}
%     \end{itemize}

%     \item<+-> Happy datatype generic programming!
%   \end{itemize}

%   \note{
%     \begin{itemize}
%       \item 今回つくった\lstinline|TestObject|の全体のソースはGitHubで公開してある

%       \item Scala 2と3ではマクロ(メタプロ)の互換性がないし、shapelessも2と3に互換性がない

%       \item 一方でScala 3はdatatype generic programmingを最初からサポートしている。
%       とはいえill-typedなリストを使わなくてもいいような方法とがあればいいと思う

%       \item datatype generic programmingを楽しもう!
%     \end{itemize}
%   }
% \end{frame}

% \begingroup
% \setbeamertemplate{frametitle}[default][right]
% \setbeamercolor{frametitle}{bg=white}

% \begin{frame}
%   \frametitle{%
%     \includegraphics[height=4ex]{img/01_Study_Sapuri_English_Horizontal.png}
%   }%

%   \begin{columns}
%     \begin{column}{0.6\textwidth}
%       \begin{itemize}
%         \item As March 1st, the number of lines of Scala 2 \& 3 source code is 878,434 in our product.
%         \begin{itemize}
%           \item This does not include generated code (such as protobuf \& gRPC), so the total is approximately over 1,000,000.
%         \end{itemize}

%         \item There are many microservices (Fig.\ref{fig:graph}), making it a very complicated system\ce{:innocent:}

%         \item The number of our server-side team members is about 16.

%         \item Meta-programming, which includes not only datatype generic programming but also \emph{scalafix} and so on,
%         is very useful for us.
%       \end{itemize}
%     \end{column}
%     \begin{column}{0.4\textwidth}
%       \begin{figure}[h]
%         \includegraphics[width=\columnwidth]{img/graph.png}
%         \caption{Very complicated micro services}
%         \label{fig:graph}
%       \end{figure}
%     \end{column}
%   \end{columns}

%   \note{
%     \begin{itemize}
%       \item 我々はスタディーサプリENGLISHというプロダクトを開発していて、
%       ソースコード規模としては3月時点でテスト込みで87万行ある。
%       protobufやgRPCで生成されたコードを含めると100万行以上となる

%       \item 図\ref{fig:graph}にあるようにマイクロサービスの数も多く、関係も複雑となっている

%       \item こういうものを16名程度のメンバーで維持するとなると、今回紹介したdatatype generic programmingや
%       他にもscalafixといったメタプログラミングのテクニックはとても重要となる
%     \end{itemize}
%   }
% \end{frame}
% \endgroup

\section*{References}
\begin{frame}%[allowframebreaks]
  \frametitle{References}
  % \nocite{*}
  \bibliographystyle{junsrt_url}
  \bibliography{ref}
\end{frame}

\begin{frame}
  \centering
  {\Huge Thank you for the attention!}
\end{frame}

\end{document}
